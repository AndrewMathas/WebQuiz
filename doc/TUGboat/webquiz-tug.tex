\documentclass{ltugboat}
\author{Andrew Mathas}
\title{Webquiz: writing online quizzes with \LaTeX}
\address{School of Mathematics and Statistics, University of Sydney, Australia}

% packages
\usepackage{xparse}
\usepackage[svgnames]{xcolor}
\RequirePackage{graphicx}
\usepackage[colorlinks=true]{hyperref}

%% macros
\NewDocumentCommand\WebQuiz{s}{\textcolor{Sienna}{WebQuiz}\IfBooleanF{#1}{\space}}
\NewDocumentCommand\ctan{sO{pkg/#3}m}{\href{https://www.ctan.org/#2}{\texttt{#3}}\IfBooleanF{#1}{\space}}

\begin{document}
\maketitle
\includegraphics[width=0.48\textwidth]{ctanlion}

\begin{abstract}
 \WebQuiz is a 2019 \LaTeX\ package for writing formative online
 assessment quizzes. Using the package it is possible to write online
 quizzes using only basic knowledge of \LaTeX. Single answer and
 multiple choice questions are supported and it is possible to give
 feedback to the student based on their response.  The quizzes are
 written in \LaTeX\ and converted to HTML using a python program
 \textsf{webquiz} (and \ctan*{tex4ht}). Behind the scenes the quizzes
 are controlled by CSS and javascript.
\end{abstract}

\section{A brief history of \TeX}
Donald Knuth released \TeX\ in 1978, motivated at least in part by his
unhappiness with the typesetting of the second volume of his \textit{The
Art of Computer Programming}. It is quite amazing that forty years later
\TeX\  is still the de facto standard for writing books and papers in
technical fields such as computer science, mathematics, engineering and
physics.

\TeX\ was designed to make it easy to publish high quality printed
research articles and books without expert knowledge.  Over the last
twenty years the field of publishing has changed dramatically with many
sources now being published and read online.  Using packages like
\ctan{TeX4ht} and \ctan{lwarp}, the capabilities of \TeX\ have expanded
so that it is now possible so that \TeX\ and \LaTeX\ can generate web
pages and online material.

One of the reasons why \TeX\ has been so successful is that it gives a
relatively easy to write high quality papers and books. The advent of
web is changing how we publish because it is now possible to write
dynamic content that ``responds'' and ``interacts'' with the reader.
Such pages usually require sophisticated javascript, which puts them
beyond the capabilities of many authors. There is a need for easy to use
tools for writing dynamic online content.  \TeX\ could easily fill this
void.

\section{\WebQuiz}
The \WebQuiz

\end{document}

