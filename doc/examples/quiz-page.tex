\documentclass[pst2pdf]{mathquiz}
\usepackage[MATH1001]{sms-uos}
\usepackage{pst-all,pst-3dplot}
\title{Quiz 4: Curves and surfaces in 3-dimensional space}
\begin{document}

% Question 1
\begin{question}
What are suitable parametric equations for this plane curve?
\begin{center}\psset{unit=6mm}\begin{pspicture}(-2.5,-1.5)(5,5.5)%
    \psaxes[linecolor=red,linewidth=1pt]{->}(0,0)(-2.5,-1.5)(5,5)%
    \psellipse[linecolor=blue,linewidth=2pt](1,2)(3,2)%
\end{pspicture}\end{center}

\begin{choice}[1]
\incorrect $x=2\cos t + 1,\ y=3\sin t + 2$
\response This is an ellipse with centre $(1,2)$ and with axes of
length $4$ in the $x$--direction and $6$ in the $y$--direction.
\begin{center}\psset{unit=6mm}\begin{pspicture}(-2.5,-0.5)(5,5.5)
    \psaxes[linecolor=red,linewidth=1pt,labels=none]{->}(0,0)(-2.5,-1.5)(5,5)
    \parametricplot[linecolor=blue,linewidth=2pt]{0}{360}%
           { t cos 2 mul 1 add t sin 3 mul 2 add}
\end{pspicture}\end{center}

\correct $x=3\cos t + 1,\ y=2\sin t + 2$

\response The curve is an ellipse centre (1,2) with axes length 6 in
the $x$ direction and 4 in the $y$ direction.

\incorrect $x=3\cos t - 1,\ y=2\sin t - 2$
\response This is an ellipse with centre $(-1,-2)$ and with axes of
length $6$ in the $x$--direction and $4$ in the $y$--direction.
\begin{center}\psset{unit=6mm}\begin{pspicture}(-5,-4)(1,2)
    \psaxes[linecolor=red,linewidth=1pt,labels=none]%
            {<-}(0,0)(-4.5,-5.5)(1,2)
    \parametricplot[linecolor=blue,linewidth=2pt]{0}{360}%
           { t cos 3 mul 1 sub t sin 2 mul 2 sub}
\end{pspicture}\end{center}

\incorrect $x=2\cos t - 1,\ y=3\sin t - 2$
\response This is an ellipse with centre $(-1,-2)$ and with axes of
length $4$ in the $x$--direction and $6$ in the $y$--direction.
\begin{center}\psset{unit=6mm}\begin{pspicture}(-4,-5)(1,2)
    \psaxes[linecolor=red,linewidth=1pt,labels=none]%
            {<-}(0,0)(-4.5,-5.5)(1,2)
    \parametricplot[linecolor=blue,linewidth=2pt]{0}{360}%
           { t cos 2 mul 1 sub t sin 3 mul 2 sub}
\end{pspicture}\end{center}

\end{choice}
\end{question}

% Question 2
\begin{question}
Which of the following could be the parametric equations for the
following space curve? (You may assume that $t\ge 0$ and that the scales are the same on the $x,\,y$ and $z$ axes.)
\begin{center}\psset{unit=6mm}\begin{pspicture}(-2,-1)(5,10)
	\parametricplotThreeD[%
		xPlotpoints=200,%
		linecolor=blue,%
		linewidth=1.5pt,%
		plotstyle=curve](0,2.8){%
			t 4 mul t 360 mul cos mul
			t 4 mul t 360 mul sin mul
			t 1.5 mul %
	}
	\pstThreeDCoor[linewidth=1pt,
            xMin=-1,xMax=6,yMin=-1,yMax=6,zMin=-1,zMax=9.5]
\end{pspicture}\end{center}

\begin{choice}[1]
\incorrect $x=\cos t,\ y=\sin t,\ z=t$
\response This curve has the following graph.
\begin{center}\psset{unit=6mm}\begin{pspicture}(-2,-1)(5,7)
	\parametricplotThreeD[%
		xPlotpoints=200,%
		linecolor=blue,%
		linewidth=1.5pt,%
		plotstyle=curve](0,4.5){%
			 t 360 mul cos
			 t 360 mul sin
			t %
	}
	\pstThreeDCoor[linewidth=1pt,
              xMin=-1,xMax=6,yMin=-1,yMax=6,zMin=-1,zMax=8]
\end{pspicture}\end{center}

\incorrect $x=t \cos t,\ y=t \sin t,\ z=t^2$
\response In this curve the $z$ values increase as the square of $t$, making the spiral much `steeper' than the curve shown in the question. The parametric equations given in
this option correspond to the following graph.
\begin{center}\psset{unit=6mm}\begin{pspicture}(-2,-1)(5,7)
	\parametricplotThreeD[%
		xPlotpoints=200,%
		linecolor=blue,%
		linewidth=1.5pt,%
		plotstyle=curve](0,3){%
			t t 360 mul cos mul
			t t 360 mul sin mul
			t dup mul%
	}
	\pstThreeDCoor[linewidth=1pt,
             xMin=-1,xMax=6,yMin=-1,yMax=6,zMin=-1,zMax=8]
\end{pspicture}\end{center}

\incorrect $x=\cos t,\ y=\sin t,\ z=t^{2}$
\response This curve has the following graph.
\begin{center}\psset{unit=6mm}\begin{pspicture}(-2,-1)(5,7)
	\parametricplotThreeD[%
		xPlotpoints=200,%
		linecolor=blue,%
		linewidth=1.5pt,%
		plotstyle=curve](0,3){%
			t 360 mul cos
			t 360 mul sin
			t dup mul
	}
	\pstThreeDCoor[linewidth=1pt,
             xMin=-1,xMax=6,yMin=-1,yMax=6,zMin=-1,zMax=8]
\end{pspicture}\end{center}

\correct $x=t\cos t,\ y=t\sin t,\ z=t$

\response The curve circles around the $z$-axis and the distance of
the curve from the $z$-axis increases \emph{linearly} with $t$ (and
hence $z$).

\end{choice}
\end{question}

% Question 3
\begin{question}
Which of the following could be the parametric equations for the
following space curve?
\begin{center}\psset{unit=12mm}\begin{pspicture}(-2,-3)(5,5)
\psdot(0,-.8)\psdot(0,.8)\psdot(0.7,0.38)
	\parametricplotThreeD[%
		xPlotpoints=200,%
		linecolor=blue,%
		linewidth=1.5pt,%
		plotstyle=curve](-1.5,2){%
			t dup mul 1 sub
			t 3 exp t sub
			t 3 exp %
	}
	\pstThreeDCoor[linewidth=1pt,xMin=-2,xMax=4,yMin=-2,yMax=4,zMin=-2,zMax=5]
\end{pspicture}\end{center}
(Hint: look at the sign of $y$ as $t$ changes.)

\begin{choice}
\correct $x=t^{2}-1,\ y=t^{3}-t,\ z=t^{3}$

\response  As $t$ increases from large negative values through zero to large positive values, $y$ starts out negative then becomes
positive, then negative and finally positive again. This fits the curve shown. It is actually quite hard to ``see'' where in space this curve lies, from the diagram provided,
so don't worry if you didn't
get this one correct.

\incorrect $x=\cos t,\ y=\sin t,\ z=t^{3}$
\response
The $\cos t$ and $\sin t$ terms make this curve wind around the $z$--axis,
and the $z=t^3$ term makes the height of the curve above the $xy$--plane
increase rapidly.
\begin{center}\psset{unit=8mm}\begin{pspicture}(-2,-1)(5,3)
	\parametricplotThreeD[%
		xPlotpoints=200,%
		linecolor=blue,%
		linewidth=1.5pt,%
		plotstyle=curve](-1.2,1.6){%
			t 360 mul cos
			t 360 mul sin
			t 3 exp %
	}
	\pstThreeDCoor[linewidth=1pt,xMin=-2,xMax=4,yMin=-2,yMax=4,zMin=-2,zMax=5]
\end{pspicture}\end{center}

\incorrect $x=t^{2}-1,\ y=t^{2},\ z=t^{2}$
\response This is a ``half'' straight line because $z=y=x+1$, for all
$t$. (It is only half a line because $x\ge-1$, $y\ge0$ and $z\ge0$
since $t^2\ge0$ for all $t$.)
\begin{center}\psset{unit=8mm}\begin{pspicture}(-2,-4)(5,3)
        \psset{Alpha=80,Beta=40}
        {\psset{linewidth=1pt,linecolor=gray,linestyle=dotted}
         \pstThreeDBox(0,0,0)(0,0,4)(3,0,0)(0,4,0)
        }
	\parametricplotThreeD[%
		xPlotpoints=200,%
		linecolor=blue,%
		linewidth=2pt,%
		plotstyle=curve](0,2.5){%
			t dup mul 1 sub
			t dup mul
			t dup mul %
	}
\       \rput[l](3,1){\blue\boldmath $t=2 \equiv (3,4,4)$}
        \pstThreeDDot(3,4,4)
	\pstThreeDCoor[linewidth=1pt,xMin=-2,xMax=4,yMin=-2,yMax=4,zMin=-2,zMax=4]
\end{pspicture}\end{center}
(The box is there  to help visualize the line).

\incorrect $x=t^{3}-t,\ y=t^{2}-1,\ z=t^{2}$
\response
\begin{center}\begin{pspicture*}(-4,-3)(5,4)
     \psset{unit=12mm,Alpha=-10,Beta=10}
     \parametricplotThreeD[%
		xPlotpoints=200,%
		linecolor=blue,%
		linewidth=1.5pt,%
		plotstyle=curve](-1.5,2){%
			t 3 exp t sub
			t dup mul 1 sub
			t dup mul %
	}
	\pstThreeDCoor[linewidth=1pt,
           xMin=-2,xMax=8,yMin=-2,yMax=4,zMin=-2,zMax=5]
\end{pspicture*}\end{center}

\end{choice}
\end{question}

% Question 4
\begin{question}
Let $f(x,y)=x^{2}+y^{2}-1$.
Which of the following could be the graph of $z=f(x,y)$?
\begin{choice}
\incorrect\begin{center}\begin{pspicture}(-2,-2)(5,5)
	\psset{unit=6mm}
	\psplotThreeD[%
		linecolor=blue,%
		plotstyle=curve,%
		drawStyle=yLines,% is the default anyway
		yPlotpoints=50,xPlotpoints=50,%
		linewidth=1pt](-1.6,1.6)(-1.6,1.6){%
                1 x dup mul y dup mul add sub}
	\pstThreeDCoor[linewidth=1pt,xMin=-5,xMax=5,yMin=-5,yMax=5,
	               zMin=-2,zMax=3]
\end{pspicture}\end{center}
\response This
cannot be the graph of $f(x,y)$. Look again at the formula for $f(x,y)$! It tells us that $f(x,y)\ge -1$ for all $x,y$. In fact, this option  is the graph of
$z=1-(x^2+y^2)$.

\incorrect\begin{center}\begin{pspicture}(-2,-2)(5,5)
	\psset{unit=6mm}
	\psplotThreeD[%
		linecolor=blue,%
		plotstyle=curve,%
		drawStyle=yLines,% is the default anyway
		yPlotpoints=50,xPlotpoints=50,%
		linewidth=1pt](-1.6,1.6)(-1.6,1.6){%
                x dup mul y dup mul add sqrt 2 mul 1 sub}
	\pstThreeDCoor[linewidth=1pt,xMin=-5,xMax=5,yMin=-5,yMax=5,
	               zMin=-2,zMax=5]
\end{pspicture}\end{center}
\response This is the graph of a regular cone (it has ``straight
sides''), whereas $f(x,y)$ is a paraboloid. In fact, this is the graph
of $z=\sqrt{x^2+y^2-1}$.


\incorrect\begin{center}\begin{pspicture}(-2,-2)(5,5)
	\psset{unit=6mm}
	\psplotThreeD[%
		linecolor=blue,%
		plotstyle=curve,%
		drawStyle=yLines,% is the default anyway
		yPlotpoints=50,xPlotpoints=50,%
		linewidth=1pt](-1.6,1.6)(-1.6,1.6){%
                x dup mul y dup mul add}
	\pstThreeDCoor[linewidth=1pt,xMin=-5,xMax=5,yMin=-5,yMax=5,zMin=-2,zMax=5]
\end{pspicture}\end{center}
\response Even though this is the graph of a paraboloid it cannot be
the graph of $z=f(x,y)$ since $f(0,0)=-1$ whereas this graph goes
through $(0,0,0)$. In fact, this is the graph of $z=x^2+y^2$.


\correct\begin{center}\begin{pspicture}(-2,-2)(5,5)
	\psset{unit=6mm}
	\psplotThreeD[%
		linecolor=blue,%
		plotstyle=curve,%
		drawStyle=yLines,% is the default anyway
		yPlotpoints=50,xPlotpoints=50,%
		linewidth=1pt](-1.6,1.6)(-1.6,1.6){%
                x dup mul y dup mul add 1 sub}
	\pstThreeDCoor[linewidth=1pt,xMin=-5,xMax=5,yMin=-5,yMax=5,
	               zMin=-2,zMax=5]
\end{pspicture}\end{center}
\response This is the only graph listed which is paraboloid passing
through $(0,0,-1)$.

\end{choice}
\end{question}

% Question 5
\begin{question}
Which option could be the equation of the following surface?
\begin{center}\begin{pspicture}(-2,-2)(5,5)
	\psset{unit=8mm}
	\psplotThreeD[%
		linecolor=blue,%
		plotstyle=line,%
		drawStyle=xLines,% is the default anyway
		yPlotpoints=25,xPlotpoints=25,%
		linewidth=1pt](-2,2)(-3,3){%
		x dup mul neg 4 add}
	\pstThreeDCoor[linewidth=1pt,xMin=-5,xMax=5,yMin=-5,yMax=5,zMin=-2,zMax=5]
\end{pspicture}\end{center}
\begin{choice}
\incorrect $z=4-x^{2}-y^{2}$
\response The function $f(x,y)=4-(x^2+y^2)$ is an inverted paraboloid.
It has the following graph.
\begin{center}\begin{pspicture}(-2,-2)(5,5)
	\psset{unit=8mm}
	\psplotThreeD[%
		linecolor=blue,%
		plotstyle=curve,%
		drawStyle=yLines,% is the default anyway
		yPlotpoints=50,xPlotpoints=50,%
		linewidth=1pt](-1.6,1.6)(-1.6,1.6){%
		x dup mul y dup mul add neg 4 add}
	\pstThreeDCoor[linewidth=1pt,xMin=-5,xMax=5,yMin=-5,yMax=5,zMin=-2,zMax=5]
\end{pspicture}\end{center}

\incorrect $z=y^{2}-2$
\response This function does not depend on $x$. Its graph is:
\begin{center}\begin{pspicture}(-2,-2)(5,5)
	\psset{unit=8mm}
	\psplotThreeD[%
		linecolor=blue,%
		plotstyle=line,%
		drawStyle=yLines,%
		yPlotpoints=25,xPlotpoints=25,%
		linewidth=1pt](-2,2)(-2,2){%
		y dup mul 2 sub}
	\pstThreeDCoor[linewidth=1pt,xMin=-5,xMax=5,yMin=-5,yMax=5,zMin=-1,zMax=5]
\end{pspicture}\end{center}

\incorrect $z=-x^{2}$
\response This surface lies entirely below the $xy$--plane so
it cannot be the function pictured. Its graph is
\begin{center}\begin{pspicture}(-5,-5)(1,1)
	\psset{unit=8mm}
	\psplotThreeD[%
		linecolor=blue,%
		plotstyle=curve,%
		drawStyle=xLines,% is the default anyway
		yPlotpoints=50,xPlotpoints=50,%
		linewidth=1pt](-2,2)(-3,3){%
		x dup mul neg }
	\pstThreeDCoor[linewidth=1pt,xMin=-5,xMax=5,yMin=-5,yMax=5,zMin=-5,zMax=1]
\end{pspicture}\end{center}

\correct $z=4-x^{2}$

\response This graph has a parabolic cross-section $z=-x^{2}$
which does not depend on the value of $y$.

\end{choice}
\end{question}

% Question 6
\begin{question}
Which of the following equations could describe this surface?
\begin{center}\psset{unit=2mm}\begin{pspicture}(-4,-12)(4,12)
  \psplotThreeD[%
		linecolor=blue,%
     plotstyle=ecurve,%
     drawStyle=yLines,%
     yPlotpoints=50,xPlotpoints=50,
     linewidth=1pt](-20,20)(-20,20){%
     x dup mul y dup mul add sqrt 45 mul cos 2 mul }
   \pstThreeDCoor[linewidth=1pt,
       xMin=-30,xMax=30,yMin=-30,yMax=30,zMin=-3,zMax=20]
  \end{pspicture}
\end{center}

\begin{choice}
\incorrect $z=\cos x \cos y$
\response
This surface looks like an egg carton, as shown below. %In particular, it is not
%symmetric in $x^2+y^2$ so it cannot be the surface described by the equation in this option.
The surface in the question can't be given by the equation $z=\cos x\cos y$ because in the plane $x=y$, for example, we have $z=\cos^2 x$ and so $z\ge 0$, which is not the case.
\begin{center}\psset{unit=2mm}
  \begin{pspicture}(-4,-12)(4,12)
  \psplotThreeD[%
		linecolor=blue,%
     plotstyle=curve,%
     drawStyle=yLines,%
     yPlotpoints=80,xPlotpoints=80,
     linewidth=1pt](-20,20)(-20,20){%
     x 45 mul cos y 45 mul cos mul 1.5 mul}
   \pstThreeDCoor[linewidth=1pt,xMin=-30,xMax=30,yMin=-30,yMax=30,zMin=-2,zMax=20]
  \end{pspicture}
\end{center}

\incorrect $z=\sin x \sin y$
\response
As $\sin x\sin y=\cos(x+\frac\pi 2)\cos(y+\frac\pi2)$ this surface is
just a shifted version of the surface $z=\cos x\cos y$.
\begin{center}\psset{unit=2mm}\begin{pspicture}(-4,-12)(4,12)
  \psplotThreeD[%
		linecolor=blue,%
     plotstyle=curve,%
     drawStyle=yLines,%
     yPlotpoints=80,xPlotpoints=80,
     linewidth=1pt](-20,20)(-20,20){%
     x 45 mul sin y 45 mul sin mul 1.5 mul}
   \pstThreeDCoor[linewidth=1pt,xMin=-30,xMax=30,yMin=-30,yMax=30,zMin=-2,zMax=20]
  \end{pspicture}
\end{center}

\incorrect $z=\cos x \sin y$
\response As $\cos x\sin y=\cos(x)\cos(y+\frac\pi2)$ this surface is
just another shifted version of the surface $z=\cos x\cos y$.
\begin{center}\psset{unit=2mm}\begin{pspicture}(-4,-12)(4,12)
  \psplotThreeD[%
		linecolor=blue,%
     plotstyle=curve,%
     drawStyle=yLines,%
     yPlotpoints=80,xPlotpoints=80,
     linewidth=1pt](-20,20)(-20,20){%
     x 45 mul cos y 45 mul cos mul 1.5 mul}
   \pstThreeDCoor[linewidth=1pt,xMin=-30,xMax=30,yMin=-30,yMax=30,zMin=-2,zMax=20]
  \end{pspicture}
\end{center}
\incorrect $z=\cos(x^2+y^2)$
\response This function looks like it could be the
graph in the question. However, the ripples in this function get
closer together as $\sqrt{x^2+y^2}$ gets large, so it cannot be the
right answer.
\begin{center}\psset{unit=8mm}\begin{pspicture}(-4,-6)(4,6)
  \psplotThreeD[%
		linecolor=blue,%
     plotstyle=ecurve,%
     drawStyle=yLines,%
     yPlotpoints=80,xPlotpoints=80,
     linewidth=1pt](-6,6)(-6,6){%
     x dup mul y dup mul add 45 mul cos 2 div}
   \pstThreeDCoor[linewidth=1pt,xMin=-8,xMax=8,yMin=-8,yMax=8,zMin=-2,zMax=6]
  \end{pspicture}
\end{center}

\correct $z=\cos\sqrt{x^{2}+y^{2}}$
\response This matches the picture and in fact is the surface we obtain by  rotating $z=\cos x$ about the
$z$-axis.

\incorrect  $z=\sin\sqrt{x^{2}+y^{2}}$
\response This is very similar to the graph in the question; however,
this cannot be the right function because $z=0$ when $(x,y)=(0,0)$
which does not agree with the graph in the question.
\begin{center}\psset{unit=2mm}\begin{pspicture}(-4,-6)(4,20)
  \psplotThreeD[%
		linecolor=blue,%
     plotstyle=ecurve,%
     drawStyle=yLines,%
     yPlotpoints=50,xPlotpoints=50,
     linewidth=1pt](-20,20)(-20,20){%
     x dup mul y dup mul add sqrt 45 mul sin 2 mul }
   \pstThreeDCoor[linewidth=1pt,
       xMin=-30,xMax=30,yMin=-30,yMax=30,zMin=-3,zMax=20]
  \end{pspicture}
\end{center}

\end{choice}
\end{question}

% Question 7
\begin{question}
Which of the equations below could describe the following surface?
\begin{center}
\psset{unit=6mm}
  \begin{pspicture*}(-6,-3)(6,6)
     \pstThreeDCoor[linewidth=1pt,%
       xMin=-8,xMax=8,yMin=-8,yMax=8,zMin=-2,zMax=6]
     \psplotThreeD[%
       linecolor=blue,%
       %plotstyle=ecurve,%
       drawStyle=yLines,%
       yPlotpoints=80,xPlotpoints=31,
       linewidth=1pt](-4,4)(-4,4)%
       {1 x dup mul y dup mul add div }
  \end{pspicture*}
\end{center}


\begin{choice}
\incorrect \(\displaystyle z=\log(x^{2}+y^{2}) \)
\response This cannot be the right graph because
\(\displaystyle z\longrightarrow-\infty \) as \(\displaystyle (x,y)\longrightarrow(0,0) \).
In fact, the function \(\displaystyle z=\log(x^{2}+y^{2}) \) has the graph:
\begin{center}\psset{unit=6mm}\begin{pspicture*}(-6,-6)(6,6)
  \psplotThreeD[%
		linecolor=blue,%
     plotstyle=line,%
     drawStyle=yLines,%
     yPlotpoints=40,xPlotpoints=40,
     linewidth=1pt](-4,4)(-4,4){%
     x dup mul y dup mul add log }
   \pstThreeDCoor[linewidth=1pt,
     xMin=-8,xMax=8,yMin=-8,yMax=8,zMin=-2,zMax=6]
  \end{pspicture*}
\end{center}

\incorrect \(\displaystyle z=e^{x^{2}+y^{2}} \)
\response This function has a very ``steep'' surface. It cannot be the function in the question
because $z=1$ when $(x,y)=(0,0)$.
In fact, \(\displaystyle z=e^{x^{2}+y^{2}} \) has the following
graph.
\begin{center}\psset{unit=7mm}\begin{pspicture*}(-4,-4)(4,4)
  \psplotThreeD[%
     linecolor=blue,%
     plotstyle=curve,%
     drawStyle=yLines,%
     yPlotpoints=30,xPlotpoints=30,
     linewidth=1pt](-4,1)(-4,1){%
     1.52 x dup mul y dup mul add exp  }
   \pstThreeDCoor[linewidth=1pt,
     xMin=-4,xMax=4,yMin=-4,yMax=4,zMin=-2,zMax=6]
  \end{pspicture*}
\end{center}

\incorrect \(\displaystyle z=1-e^{x^{2}+y^{2}} \)
\response This function has a very ``steep'' inverted surface. It cannot be the function in this option
because here, $z$ is always less than 1 and becomes very large and negative as $x^2+y^2$ increases. In fact,
\(\displaystyle z=1-e^{x^{2}+y^{2}} \) has the following
graph.
\begin{center}\psset{unit=6mm}
  \begin{pspicture*}(-4,-4)(4,4)
  \psplotThreeD[%
		linecolor=blue,%
     plotstyle=line,%
     drawStyle=yLines,%
     yPlotpoints=30,xPlotpoints=30,
     linewidth=1pt](-4,4)(-4,4){%
     1.52 x dup mul y dup mul add exp neg 1 add}
   \pstThreeDCoor[linewidth=1pt,
       xMin=-4,xMax=4,yMin=-4,yMax=4,zMin=-2,zMax=6]
  \end{pspicture*}
\end{center}

\correct \(z=\dfrac1{x^2+y^2}\)
\response This surface is the
only function listed which is consistent with the graph in the
question, as
$\frac1{x^{2}+y^{2}}\rightarrow\infty$ as $(x,y)\rightarrow(0,0).$

\end{choice}
\end{question}

% Question 8
\begin{question}
    Which one of the following shows some of the
    level curves for the function $z=x(x+y)$ ?
\begin{choice}
    \incorrect
     \begin{center}\psset{unit=4mm}
       \begin{pspicture}(-3.5,-4.5)(5,8.5)
         \psaxes[linecolor=red,linewidth=1pt,labels=none]%
                {->}(0,0)(-2.5,-4.5)(5,8)
         \psellipse[linecolor=blue,linewidth=2pt](1,2)(1.5,1)
         \psellipse[linecolor=blue,linewidth=2pt](1,2)(3,2)
         \psellipse[linecolor=blue,linewidth=2pt](1,2)(4.5,3)
        \end{pspicture}\end{center}
	\response These level curves are ellipses with equations of the form $\frac{(x-1)^2}{a^2}+\frac{(y-2)^2}{b^2}=1$, which are not the same equations as the level curves
	 for the
	function with equation $z=x(x+y)$.

    \correct
     \begin{center}\psset{unit=4mm}
        \begin{pspicture}(-2.5,-4.5)(5,6.5)
          \psaxes[linecolor=red,linewidth=1pt,labels=none]%
                {->}(0,0)(-5.5,-4.5)(5.5,6)
          \psplot[linecolor=blue,plotstyle=curve,plotpoints=200,%
                            linewidth=2pt]%
	       {-4}{-0.5}{1 x div x sub}
          \psplot[linecolor=blue,plotstyle=curve,plotpoints=200,%
                            linewidth=2pt]%
	       {0.5}{4}{1 x div x sub}
          \psplot[linecolor=blue,plotstyle=curve,plotpoints=200,%
                            linewidth=2pt]%
	       {-4}{-0.5}{2 x div x sub}
          \psplot[linecolor=blue,plotstyle=curve,plotpoints=200,%
                            linewidth=2pt]%
	       {0.5}{4}{3 x div x sub}
          \psplot[linecolor=blue,plotstyle=curve,plotpoints=200,%
                            linewidth=2pt]%
	       {-4}{-0.5}{3 x div x sub}
          \psplot[linecolor=blue,plotstyle=curve,plotpoints=200,%
                            linewidth=2pt]%
	       {0.5}{4}{2 x div x sub}
	  \psline[linecolor=green,linestyle=dotted,linewidth=2pt]%
	     (2,0)(2,0)
        \end{pspicture}\end{center}
    \response The level curve at height $z=c$ satisfies the
    equation $c=x(x+y)$; that is,
    \(\displaystyle y=\frac cx-x \).

   \incorrect
     \begin{center}\psset{unit=4mm}
        \begin{pspicture}(-2.5,-4.5)(5,6.5)
	  \psline[linecolor=green,linestyle=dotted,linewidth=2pt]%
	     (2,-5)(2,5)
          \psaxes[linecolor=red,linewidth=1pt,labels=none]%
                {->}(0,0)(-5.5,-4.5)(5.5,6)
          \psplot[linecolor=blue,plotstyle=curve,plotpoints=200,%
                            linewidth=2pt]%
	       {-4}{1.5}{1 x 2 sub div x 2 sub sub}
          \psplot[linecolor=blue,plotstyle=curve,plotpoints=200,%
                            linewidth=2pt]%
	       {2.5}{5}{1 x 2 sub div x 2 sub sub}
          \psplot[linecolor=blue,plotstyle=curve,plotpoints=200,%
                            linewidth=2pt]%
	       {-4}{1.5}{2 x 2 sub div x 2 sub sub}
          \psplot[linecolor=blue,plotstyle=curve,plotpoints=200,%
                            linewidth=2pt]%
	       {2.5}{5}{3 x 2 sub div x 2 sub sub}
          \psplot[linecolor=blue,plotstyle=curve,plotpoints=200,%
                            linewidth=2pt]%
	       {-4}{1.5}{3 x 2 sub div x 2 sub sub}
          \psplot[linecolor=blue,plotstyle=curve,plotpoints=200,%
                            linewidth=2pt]%
	       {2.5}{5}{2 x 2 sub div x 2 sub sub}
        \end{pspicture}\end{center}
    \response These level curves appear to have an asymptote at $x=2$
    whereas the level curves of $z=f(x,y)$ have an asymptote at $x=0$.

    \incorrect
     \begin{center}\psset{unit=4mm}
        \begin{pspicture}(-2.5,-4.5)(5,8.5)
          \psaxes[linecolor=red,linewidth=1pt,labels=none]%
                  {->}(0,0)(-2.5,-4.5)(8,5)
          \psellipse[linecolor=blue,linewidth=2pt](1,2)(1,1.5)
          \psellipse[linecolor=blue,linewidth=2pt](1,2)(2,3)
          \psellipse[linecolor=blue,linewidth=2pt](1,2)(3,4.5)
        \end{pspicture}\end{center}
	\response These level curves are ellipses with equations of the form $\frac{(x-1)^2}{a^2}+\frac{(y-2)^2}{b^2}=1$, which are not the same equations as the level curves
	 for the
	function with equation $z=x(x+y)$.

    \end{choice}
\end{question}

% Question 9
\begin{question}
    Some of the level curves of an unknown function $z=f(x,y)$ are given
    below.
    \begin{center}\psset{unit=4mm}
       \begin{pspicture}(-5,-2.5)(8,6.5)
         \psaxes[linecolor=red,linewidth=1pt,labels=none]%
                {->}(0,0)(-5.5,-1.5)(6.5,6)
         \psellipse[linecolor=blue,linewidth=2pt](1,2)(1.5,1)
         \psellipse[linecolor=blue,linewidth=2pt](1,2)(3,2)
         \psellipse[linecolor=blue,linewidth=2pt](1,2)(4.5,3)
        \end{pspicture}\end{center}
    Which of the following functions could be $f(x,y)$ ?
    (Note: we do not know the height of these level curves.)
    \begin{choice}[1]

    \incorrect $f(x,y)=x^2+y^2$.
    \response The level curves of this function are circles centred at
    the origin.

    \incorrect $f(x,y)=4(x+1)^2+9(y+2)^2$.
    \response The level curves of this function are ellipses centered
    at $(-1,-2)$.  % and with the lengths of the $x$ and $y$ axes in the
   % ratio $2:3$.



    \incorrect $f(x,y)=9(x-1)^2+4(y-2)^2$.
    \response The level curves of this function are ellipses centered
    at $(1,2)$. The semi-major axis of each ellipse is vertical and the
    semi-minor axis is horizontal. That is, the ellipses are taller than they are wide, and so
     this option doesn't match the given set of curves.  %and with the lengths of the $x$ and $y$ axes in the
    % ratio $3:2$.

    \correct $f(x,y)=4(x-1)^2+9(y-2)^2$.
    \response The level curves of this function are ellipses centered
    at $(1,2)$. The semi-major axis of each ellipse is horizontal and the
    semi-minor axis is vertical. The ratio of these two axes is $ 3:2$, matching that shown on the set of curves. %and with the lengths of the $x$ and $y$ axes in the
    %ratio $3:2$.

    \end{choice}
\end{question}

% Question 10
\begin{question}
    Which of the following graphs \textit{could be} part of the set of level
    curves for some surface $z=f(x,y)$ ?
   \begin{choice}[multiple]
    \correct \begin{center}\psset{unit=4mm}
       \begin{pspicture}(-5,-2.5)(8,6.5)
         \psaxes[linecolor=red,linewidth=1pt,labels=none]%
                {->}(0,0)(-5.5,-1.5)(6.5,6)
         \psellipse[linecolor=blue,linewidth=2pt](1,2)(1.5,1)
         \psellipse[linecolor=blue,linewidth=2pt](1,2)(3,2)
         \psellipse[linecolor=blue,linewidth=2pt](1,2)(4.5,3)
        \end{pspicture}\end{center}
    \response For example, these could be some of the level curves of the
    surface given by \[f(x,y)=\sqrt{3(x-1)^2+2(y-2)^2}.\]
   % \correct \begin{center}\psset{unit=4mm}
   %    \begin{pspicture}(-5,-2.5)(8,6.5)
   %      \psaxes[linecolor=red,linewidth=1pt,labels=none]%
   %             {->}(0,0)(-5.5,-1.5)(6.5,6)
   %      \psline[linecolor=blue,linewidth=2pt](-5,-1)(5,4)
   %      \psline[linecolor=blue,linewidth=2pt](-3,4)(7,-1)
   %      \psline[linecolor=blue,linewidth=2pt](-5,2)(5,2)
   %     \end{pspicture}\end{center}
   % \response For example, these could be the level curves of the
   % surface given by \[f(x,y)=\dfrac{x-1}{y-2}\].
   \correct \begin{center}\psset{unit=4mm}
       \begin{pspicture}(-5,-2.5)(8,6.5)
         \psaxes[linecolor=red,linewidth=1pt,labels=none]%
                {->}(0,0)(-5.5,-1.5)(6.5,6)
         \psplot[linecolor=blue,plotstyle=curve,plotpoints=200,%
	   linewidth=2pt]{-3}{5}{3 x sub}
         \psplot[linecolor=blue,plotstyle=curve,plotpoints=200,%
	   linewidth=2pt]{-3}{5}{1 x add}
         \psplot[linecolor=blue,plotstyle=curve,plotpoints=200,%
	   linewidth=2pt]{-3}{5}{2 x 1 sub dup mul 1 add sqrt add}
         \psplot[linecolor=blue,plotstyle=curve,plotpoints=200,%
	   linewidth=2pt]{-3}{5}{2 x 1 sub dup mul 2 add sqrt add}
        \end{pspicture}\end{center}
    \response  For example, these could be some of the level curves of the
      surface given by \[f(x,y)=(x-1)^2-(y-2)^2.\]

    \correct \begin{center}\psset{unit=4mm}
       \begin{pspicture}(-5,-4.5)(8,6.5)
         \psaxes[linecolor=red,linewidth=1pt,labels=none]%
                {->}(0,0)(-5.5,-3.5)(6.5,6)
         \psplot[linecolor=blue,plotstyle=curve,plotpoints=200,%
	   linewidth=2pt]{-3}{5}{x}
         \psplot[linecolor=blue,plotstyle=curve,plotpoints=200,%
	   linewidth=2pt]{-3}{5}{1 x add}
         \psplot[linecolor=blue,plotstyle=curve,plotpoints=200,%
	   linewidth=2pt]{-3}{5}{2 x add}
         \psplot[linecolor=blue,plotstyle=curve,plotpoints=200,%
	   linewidth=2pt]{-3}{5}{3 x add}
        \end{pspicture}\end{center}
    \response  For example, these could be some of the level curves of the
      surface given by \[f(x,y)=y-x.\]


  \end{choice}
\end{question}

\end{document}


